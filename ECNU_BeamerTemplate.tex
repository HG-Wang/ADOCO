\documentclass[10pt,aspectratio=169]{beamer}		
%设置为 Beamer 文档类型,设置字体为 10pt,长宽比为16:9,数学字体为 serif 风格

%%%%-----导入宏包-----%%%%
\usepackage{ecnu}			%导入 CCNU 模板宏包
\usepackage{ctex}			 %导入 ctex 宏包,添加中文支持
\usepackage{amsmath,amsfonts,amssymb,bm}   %导入数学公式所需宏包
\usepackage{color}			 %字体颜色支持
\usepackage{graphicx,hyperref,url}	
\usepackage{booktabs} % 用于三线表
\usepackage{algorithm}
\usepackage{algorithmic}
%%%%%%%%%%%%%%%%%%	

% 仅数学使用衬线字体,替代过时的 mathserif 选项
\usefonttheme[onlymath]{serif}

% PDF 元数据,避免 \ 在 PDF 字符串中的警告
\hypersetup{
  pdfauthor={Anonymous Authors},
  pdftitle={Asynchronous Decentralized Online Convex Optimization}
}

% 安全插图:若图片不存在则显示占位框,避免编译报错
% 安全显示文件名(避免下划线等特殊字符导致数学模式错误)
\newcommand{\filenamefmt}[1]{\texttt{\detokenize{#1}}}

\newcommand{\maybeincludegraphics}[2][]{%
  \IfFileExists{#2}{%
    \includegraphics[#1]{#2}%
  }{%
    {\setlength{\fboxsep}{2pt}%
    \fbox{\parbox[c][0.45\linewidth][c]{\linewidth}{\centering{\color{red} Missing image: \filenamefmt{#2}}}}}%
  }%
}


\beamertemplateballitem		%设置 Beamer 主题

%%%%------------------------%%%%%
% 如果用 pdflatex 编译,CTeX 可能尝试使用 mac 字体集或生成位图字体,
% 并且 pdflatex 对于含有空格/非 ASCII 的文件名支持薄弱。
% 因此强烈建议使用 XeLaTeX 或 LuaLaTeX(xelatex 或 lualatex)。
\RequirePackage{iftex}
\ifPDFTeX
  \PackageError{ECNU_BeamerTemplate}{This template requires XeLaTeX or LuaLaTeX for proper Unicode/CJK and file name handling. Please run `xelatex ECNU_BeamerTemplate.tex` (or `lualatex`) instead of `pdflatex`.}{}
\fi

% 将中文句号转换为指定符号(可选)。使用更稳健的 catcode/定义形式。
\catcode`\。=\active
\def。{.}
% 将正文中的"。"号转换为"."(全角句点),如需半角请改为 \def。{.}
%%%%%%%%%%%%%%%%%%%%%

%%%%----首页信息设置----%%%%
\title[Asynchronous Decentralized Online Learning ]{Asynchronous Decentralized Online Learning}
\subtitle{Published in NeurIPS 2021}			
%%%%----标题设置

\author[J. Jiang et al.]{
  Jiyan Jiang, Wenpeng Zhang, Jinjie Gu, Wenwu Zhu \\
  \vspace{0.3cm}
  \small {Presenters: Haiguang Wang, Yufeng Chen, Yuxiang Hong}
}
%%%%----个人信息设置


\date{2025年11月7日}
%%%%----日期信息

\begin{document}

\begin{frame}
\titlepage
\end{frame}				%生成标题页

\section{Outline}
\begin{frame}
\frametitle{Outline}
\tableofcontents
\end{frame}				%生成提纲页

\renewcommand{\figurename}{image}

\section{Introduction}
\begin{frame}
  \frametitle{Motivation and Background}
  \begin{itemize}
    \item \textbf{Online learning}: Process large-scale streaming data efficiently
    \item \textbf{Existing decentralized algorithms}: Operate under synchronous setting
    \begin{itemize}
      \item Leads to \textcolor{red}{straggler problem}: Fast learners wait for slow learners
    \end{itemize}
    \item \textbf{This paper's purpose}: Design asynchronous counterpart of synchronous algorithms
  \end{itemize}
  
  \vspace{0.3cm}
  \textbf{Key challenges:}
  \begin{itemize}
    \item Each learner has five basic types of actions: prediction, weight update, message sending, message receiving, and model averaging.
    \item Asynchronization: Each learner executes its own actions independently $\Rightarrow$ the orders of the actions of different learners are intermixed together.
  \end{itemize}
\end{frame}

\begin{frame}
  \frametitle{Contributions}
  \textbf{Novel Framework}: Asynchronous Decentralized Online Convex Optimization (AD-OCO)
  \begin{itemize}
    \item First formal framework that gives a complete characterization of the whole interaction dynamics of asynchronous decentralized online learning.
    \item Individual view: Specify the action sequence of each learner using its own indices.
    \item System view: Propose a novel event indexing system $\Rightarrow$ mapping all learners into a single time axis.
  \end{itemize}
  
  \vspace{0.3cm}
  \textbf{AD-OGP Algorithm}:
  \begin{itemize}
    \item A gossip-based communication scheme
    \begin{itemize}
        \item Each learner is only allowed to communicate with its immediate neighbors.
        \item Adopts "push-sum" strategy (a common way to realize asymmetric gossiping), which does not require any local synchronization among learners.
    \end{itemize}
    \item A local update scheme build upon online gradient descent (OGD).
    \begin{itemize}
        \item Design a novel \textcolor{blue}{weighted projection} operation to avoid predictions outside the feasible domain.
    \end{itemize}
    \item \textcolor{blue}{Instantaneous model averaging}: Each learner performs model averaging as long as it's unoccupied and its receiving buffer is not empty.
  \end{itemize}
\end{frame} 

\begin{frame}

  \textbf{Theoretical Contributions}:
  \begin{itemize}
    \item Conduct the \textbf{first} regret analysis of asynchronous decentralized online learning.
    \item Extend the graph augmentation technique to online setting to handle message delays.
    \item Disentangle the complex effects of predictions, weight update and model averaging via our proposed event indexing system.
    \item Derive a non-trivial regret bound of AD-OGP: it is the same order $O(\sqrt{T})$ as that of its synchronous counterpart.
    \item Make an extra contribution to the convergence analysis of push-sum: reduce a factor of $\sqrt{n}$ (n is the dimension of model parameters) in the convergence rate of push-sum compared to the previous results, and the final bound is now independent of n.
  \end{itemize}
\end{frame}

\section{AD-OCO Framework}
\begin{frame}
  \frametitle{Problem Formulation}
  \textbf{Network Structure}:
  \begin{itemize}
    \item Undirected graph $\mathcal{G} = (V,E)$ with vertex set $V = \{1,\ldots,m\}$
    \item Each learner $i \in V$ maintains a local model $w_i \in \mathcal{K}$ (convex compact decision set)
    \item Each learner $i$ can only communicate with its immediate neighbors $\mathcal{N}(i) = \{j \in V \mid (i,j) \in E\}$
  \end{itemize}
  
  \vspace{0.3cm}
  \textbf{Individual View}: Specifies the action sequence of each individual learner.
  \begin{itemize}
    \item Learner $i \in V$ is selected adversarially by the environment to make predictions at certain time point.
    \item Index the prediction rounds of learner $i$ as $\tau = 1,\ldots,N_i$, learner $i$ predicts with its most recent model $w_i^\tau$.
    \item Environment reveals a convex loss function $f_i^\tau : \mathcal{K} \mapsto \mathbb{R}$, and learner suffers from the loss $f_i^\tau(w_i^\tau)$.
    \item After finishing the calculation, it uses the gradient to update its model $w_i \leftarrow \mathcal{U}(w_i;\nabla f_i^\tau(w_i^\tau))$, where $\mathcal{U}$ is the local update scheme.
  \end{itemize}
\end{frame}
\begin{frame}
  \begin{itemize}
    \item Then it selects a subset of neighbors, and \textit{sends} a copy of its updated model to each of them.
    \item Besides sending out its own model copies, learner $i$ will also \textit{receive} its neighbors' model copies at certain time points. These copies will be stored in receiving buffer $\mathcal{B}_i$.
    \item Whenever learner $i$ is not occupied with gradient calculation or local update, and its buffer is not empty, it can \textit{average} its model with the model copies stored in the buffer, i.e., $w_i \leftarrow \mathcal{A}(w_i;\mathcal{B}_i)$, where $\mathcal{A}$ is the model averaging scheme.
  \end{itemize}
\end{frame}

\begin{frame}
  \frametitle{System View and Event Indexing}
  \textbf{Global Event Sequence}: Due to asynchronization, each learner executes its actions independently. $\Rightarrow$ The order of actions of all learners are intermixed.
  \begin{itemize}
    \item We need to specify the complete orders of the actions of all learners by mapping them into a single time axis.
    \item We view either a \textbf{predictions} or a \textbf{local update} of any learner as an event, and use a virtual counter $t$ to index these events.
    \item Let $T = \sum_{i\in V} N_i$ be the total number of \textbf{predictions} of all learners, then the total number of \textbf{local updates} is also T.
    \item For each event $t \in \{1,\ldots,2T\}$, we use $\delta_t \in \{0,1\}$ to distinguish the event type:
    \begin{itemize}
      \item $\delta_t = 0$ if it is a prediction event.
      \item $\delta_t = 1$ if it is a local update event.
    \end{itemize}

    \item We further use $i_t$ to denote the learner that executes the event $t$, and $f_t$ to denote the associated feedback.
    \item Then there exists some $\tau \in \{1,\ldots,N_{i_t}\}$ such that $f_t = f_{i_t}^\tau$.
  \end{itemize}
\end{frame}

\begin{frame}
  \begin{figure}[h!]
    \centering
    \maybeincludegraphics[width=0.95\textwidth]{image1.png}
    \caption{An example of Prediction and local update.}
  \end{figure}
\end{frame}

\begin{frame}
  \frametitle{Delays and Model Evolution}
  \textbf{Processing delay}: Since any prediction must occur before its corresponding local update, here we have $t \ge 2$, $l_t < t$ and $d^g(t)>0$.
  \[
  d^p(t) = \text{card}\{l_t < s < t \mid \delta_s = 1\}
  \]
  \textit{i.e., the number of local updates between prediction and its associated local update.}
  
  \vspace{0.3cm}
  \textbf{Message delay}: After any local update event $t$, the learner $i_t$ that executes this event will send out its model copy to a subset $S_t$ of its neighbors $\mathcal{N}(i_t)$. We focus on the copy sent to any neighbor $j \in S_t$. We assume such copy is used to average learner $j$'s model between events $r_{tj}$ and ($r_{tj}+1$) for some $r_{tj} \ge t$. For the message sent from learner $i_t$ to learner $j$ after event $t$, we introduce a quantity termed the message delay $d_{i_t}^m(t) = r_{tj}-t$.

\end{frame}
\begin{frame}
  \begin{figure}[h!]
    \centering
    \maybeincludegraphics[width=0.9\textwidth]{image2.png}
    \caption{An example of Processing delay and Message delay.}
  \end{figure}
\end{frame}

\begin{frame}
  \frametitle{Model Evolution and Regret Definition}
  \textbf{Model Evolution}: We denote $w_i(t)$ as the most recent model before event $t$ occurs, and $u_i(t)$ as the model immediately after event $t$.
  \begin{itemize}
    \item For an update event executed by learner $i_t$, its associated gradient is calculated based on learner $i_t$'s prediction using model $w_{i_t}(l_t)$ at event $l_t$:
    \[
    u_{i_t} = \mathcal{U}(w_{i_t}(t);\nabla f_{l_t}(w_{i_t}(l_t)))
    \]
    The models of other learners do not change at $i_t$'s local update event, so $u_j(t)=w_j(t)$ for $j \ne i_t$.
    \item Model averaging operation is not regarded as an event. It happens between $t$ and $t+1$ if learner $i$ is not occupied.
    \begin{itemize}
        \item Model at the beginning: $u_i(t)$
        \item Model at the end: $w_i(t+1)$
    \end{itemize}
    So, the overall effect of model averaging on learner $i$ between event $t$ and $t+1$ can be expressed as:
    \[
    \boldsymbol{w}_{i}(t+1)=
    \begin{cases}
    \boldsymbol{u}_{i}(t), & s=0 \\
    \mathcal{A}\left(\cdots\left(\mathcal{A}\left(\mathcal{A}\left(\boldsymbol{u}_{i}(t), \mathcal{B}_{i}^{1}(t)\right), \mathcal{B}_{i}^{2}(t)\right) \cdots\right), \mathcal{B}_{i}^{s}(t)\right), & s \geq 1
    \end{cases}
    \]
  \end{itemize}
  
\end{frame}
\begin{frame}

  \textbf{Regret Definition}: In AD-OCO, the regret w.r.t an arbitrary reference learner $j \in V$ is defined as:
  \[
  \text{Regret}_j=\sum_{t \in \mathcal{P}_{T}} f_{t}\left(\boldsymbol{w}_{j}(t)\right)-\sum_{t \in \mathcal{P}_{T}} f_{t}\left(\boldsymbol{w}^{*}\right)
  \]
  where $\mathcal{P}_{T}=\left\{t \in\{1, \ldots,2T\} \mid \delta_{t}=0\right\}$ denotes the set of all prediction events, and $w^* \in \operatorname{argmin}_{\boldsymbol{w} \in \mathcal{K}} \sum_{t \in \mathcal{P}_{T}} f_{t}(\boldsymbol{w})$ refers to a \textbf{fixed model} chosen with hindsight.



  \textbf{Goal}: Learners aim to make a prediction sequence $\left\{\boldsymbol{w}_{i}(t)\right\}_{i \in V, t \in \mathcal{P}_{T}}$, such that the regret of any reference learner $j$ grows sublinearly with time, i.e.,
  \[
  \lim_{T \rightarrow \infty} \operatorname{Regret}_{j} / T=0
  \]

\end{frame}
\begin{frame}
  \begin{figure}[h!]
    \centering
    \maybeincludegraphics[width=0.8\textwidth]{image3.png}
    \caption{The AD-OGP Algorithm Framework (Algorithm 1).}
  \end{figure}
\end{frame}

\section{AD-OGP Algorithm}
\begin{frame}
  \frametitle{Algorithm Overview}
  \textbf{Initial State}: Each node $i$ maintains a pair of variables $(x_i, y_i)$, and predicts with $w_{i_t}(t)=\frac{x_{i_t}(t)}{y_{i_t}(t)}$.
  \begin{itemize}
    \item $x_i$ is a \textbf{weighted model}, $y_i$ is a \textbf{weight scalar} (always $>0$). Both are used for push-sum.
    \item Initialization: $x_i = w_0 \in \mathcal{K}$ ($\mathcal{K}$ is a convex compact decision set), $y_i = 1$.
  \end{itemize}
  
  \vspace{0.3cm}
  \textbf{AD-OGP Algorithm Steps}:
  \begin{algorithm}[H]
    \caption{Asynchronous Decentralized Online Gradient-Push (AD-OGP)}
    \begin{algorithmic}[1]
      \STATE \textbf{Input:} Learning rate $\eta$, initial model $w_0 \in \mathcal{K}$
      \STATE \textbf{Initialize:} For each learner $i \in V$, $x_i \leftarrow w_0$, $y_i \leftarrow 1$. Empty receiving buffer $\mathcal{B}_i$.
      \STATE \textbf{Loop for each event $t \in \{1, \ldots, 2T\}$:}
      \STATE \quad \textbf{If} $\delta_t = 0$ (prediction event) for learner $i_t$:
      \STATE \quad \quad $w_{i_t}(t) \leftarrow x_{i_t}(t)/y_{i_t}(t)$
      \STATE \quad \quad Make prediction with $w_{i_t}(t)$, receive loss function $f_t$.
      \STATE \quad \textbf{If} $\delta_t = 1$ (local update event) for learner $i_t$:
      \STATE \quad \quad Let $l_t$ be the prediction event corresponding to $f_t$.
      \STATE \quad \quad Compute gradient $g_t = \nabla f_{l_t}(w_{i_t}(l_t))$.
      \STATE \quad \quad $\gamma_{i_t} \leftarrow 1/(|\mathcal{N}(i_t)|+1)$
      \STATE \quad \quad $x_{i_t}^{temp} \leftarrow x_{i_t}(t) - \eta g_t$
      \STATE \quad \quad $w_{i_t}^{proj} \leftarrow \Pi_{\mathcal{K}}(x_{i_t}^{temp}/y_{i_t}(t))$  \COMMENT{Weighted Projection}
      \STATE \quad \quad $x_{i_t}'(t) \leftarrow \gamma_{i_t} y_{i_t}(t) w_{i_t}^{proj}$
      \STATE \quad \quad $y_{i_t}'(t) \leftarrow \gamma_{i_t} y_{i_t}(t)$
      \STATE \quad \quad Send $(x_{i_t}'(t), y_{i_t}'(t))$ to all neighbors $j \in \mathcal{N}(i_t)$.
      \STATE \quad \textbf{Asynchronous Model Averaging (between events $t$ and $t+1$):}
      \STATE \quad \quad For each learner $i$ that is unoccupied and $\mathcal{B}_i$ is not empty:
      \STATE \quad \quad \quad For each $(x_{rcv}, y_{rcv}) \in \mathcal{B}_i$:
      \STATE \quad \quad \quad \quad $x_i \leftarrow x_i + x_{rcv}$
      \STATE \quad \quad \quad \quad $y_i \leftarrow y_i + y_{rcv}$
      \STATE \quad \quad \quad Clear $\mathcal{B}_i$.
      \STATE \quad \quad For $j \ne i_t$, $x_j(t+1) \leftarrow x_j(t)$, $y_j(t+1) \leftarrow y_j(t)$.
    \end{algorithmic}
  \end{algorithm}
\end{frame}

\begin{frame}
  \frametitle{Push-Sum Communication Strategy}
  \textbf{Challenge}: Decentralized model averaging typically involves linear averaging with neighbors. In an asynchronous and asymmetric setting, traditional doubly-stochastic averaging is not feasible.
  \begin{itemize}
    \item \textit{Why?}
    \begin{itemize}
      \item Doubly-stochastic matrices require symmetric interactions ($p_{ij} = p_{ji}$) and global synchronization ($v(t+1)=Pv(t)$).
      \item In asynchronous asymmetric gossiping, learner $i$ sends its model to $j$ without waiting for acknowledgment or $j$'s model. This makes symmetric weights impossible to negotiate.
      \item Lack of global rounds means $\sum_j p_{ij}=1$ (row stochastic) and $\sum_i p_{ij}=1$ (column stochastic) cannot be maintained simultaneously in a global snapshot.
      \item Column stochasticity ($P_{ii}=\gamma_i$, $\forall j \in \mathcal{N}(i)$, $P_{ji}=\gamma_i$) can be enforced locally for outgoing messages, but row stochasticity cannot.
    \end{itemize}
    \item \textit{How Push-sum addresses this?}
    \begin{itemize}
      \item Each learner maintains a tuple $(x_i, y_i)$. $x_i$ carries model information, $y_i$ is an auxiliary scalar.
      \item After local update, learner $i$ calculates $\gamma_i = 1/(|\mathcal{N}(i)|+1)$, updates $x'_i = \gamma_i x_i$, $y'_i = \gamma_i y_i$, and sends $(x'_i, y'_i)$ to all neighbors.
      \item Neighbors accumulate these received quantities into their $(x,y)$ pair (Instantaneous Model Averaging).
      \item The predicted model is $w_i = x_i/y_i$.
      \item \textbf{Bias Removal}: While $x_i$ might converge to a biased sum $\pi_i \bar{x}$ due to network asymmetry, $y_i$ undergoes the same averaging process and converges to the same bias $\pi_i$. Thus, the ratio $w_i = x_i/y_i$ becomes unbiased, approximating the global average $\bar{x}$.
    \end{itemize}
  \end{itemize}
\end{frame}

\begin{frame}
  \frametitle{Regret Bound Analysis}
  \textbf{Assumptions}:
  \begin{itemize}
    \item \textbf{Assumption 1}: The message delays are bounded by some integer $D^{msg} \geq 0$.
    \item \textbf{Assumption 2}: Each learner performs local updates at least once every $\Gamma_d$ steps for some integer $\Gamma_d > 0$.
  \end{itemize}
  
  \vspace{0.3cm}
  \textbf{Definitions}:
  \begin{itemize}
    \item $\mathcal{Q}_{s,t} = \mathcal{Q}_T \cap (s,t)$ denotes local updates between events $s$ and $t$.
    \item For any $t \in \mathcal{Q}_T$, $d^p(t) = |\mathcal{Q}_{l_t,t}|$.
    \item $g_t = \nabla f_{l_t}(x_{i_t}(l_t)/y_{i_t}(l_t))$, $\hat{g}_t = \nabla f_{l_t}(x_j(l_t)/y_j(l_t))$ for local update gradients.
    \item Total processing delay: $D^{proc} = \sum_{t \in \mathcal{Q}_T} d^p(t)$.
  \end{itemize}
  
  \vspace{0.3cm}
  \textbf{Key Techniques for Analysis}:
  \begin{itemize}
    \item \textbf{Graph Augmentation Technique}: For each real node $i$, create $D^{msg}$ virtual nodes. Message passing from $i$ to $j$ is viewed as a delay-free path on this augmented graph. This helps to handle message delays.
    \item \textbf{Mixing Analysis}: After a sufficient number of time steps (related to network diameter $\mathcal{D}$ and max delay $D^{msg}$, denoted as $B$), the multi-step transition matrix $\mathcal{Q}_{s,s+B}$ ensures that information from any node propagates to any other real node.
    \item \textbf{Geometric Decay of Model Bias}: The minimum positive element $\alpha$ in the transition matrix is used to show that model deviation $\|w_i(t) - \bar{w}(t)\|$ decays geometrically, proportional to $\lambda^{t-s}$, where $\lambda$ is a convergence factor.
  \end{itemize}
\end{frame}

\begin{frame}
  \frametitle{Regret Bound Derivation}
  \textbf{General Regret Upper Bound}:
  \[
  \begin{array}{l}
  \text { Regret }_{j} \leq \frac{m}{2\eta}\left\|\boldsymbol{w}_{0}-\boldsymbol{w}^{*}\right\|^{2}+\frac{2\eta}{m} \sum_{t \in \mathcal{Q}_{T}} \sum_{s \in \mathcal{Q}_{l_t}, t+1} g_{t} g_{s}+\eta \sum_{t \in \mathcal{Q}_{T}} \frac{g_{t}^{2}}{y_{i_{t}}(t)} \\ \quad+2\eta \sum_{t \in \mathcal{Q}_{T}} \sum_{s \in \mathcal{Q}_{1, l_t}} \lambda^{\left\lfloor\frac{l_{t}-s}{2B}\right\rfloor}\left(g_{t}+2\hat{g}_{t}\right) \frac{g_{s}}{y_{i_{s}}(t)}+2\eta \sum_{t \in \mathcal{Q}_{T}} \sum_{s \in \mathcal{Q}_{1, t+1}} \lambda^{\left\lfloor\frac{t-s}{2B}\right\rfloor} g_{t} \frac{g_{s}}{y_{i_{s}}(t)},\end{array}
  \]
  where $\lambda=1-m \alpha^{4}$, $\alpha=(1/m)^{B}$, $B=(\mathcal{D}+1)\left(D^{msg}+\Gamma_{d}\right)$.
  
  \vspace{0.3cm}
  \textbf{Final Regret Bound} (assuming $g_t \leq G, \hat{g}_t \leq G, \forall t \in \mathcal{Q}_T$ for $G < \infty$):
  \[
  \mathbf{Regret}_j \leq \frac{m}{2\eta} \| \boldsymbol{w}_0 - \boldsymbol{w}^* \|^2 + \frac{2\eta}{m} G^2(T + D^{proc}) + \frac{2\eta}{m} \left( \frac{8 + \alpha^4}{\alpha^5} \right) BG^2T
  \]
  By setting $\eta=(mF/2G)(D^{proc}+T+CBT)^{-1/2}$, where $C=(8+\alpha^4)/\alpha^5$, the final regret is $O((CBT+T+D^{proc})^{1/2})$.
\end{frame}

\section{Experiments}
\begin{frame}
  \frametitle{Experimental Setup}
  \textbf{Two Large-Scale Real-World Datasets}:
  \begin{itemize}
    \item \textbf{Higgs dataset}: A benchmark dataset in high-energy physics for binary classification, consisting of 11 million instances with 28 features. Uses logical loss.
    \item \textbf{Poker-hand dataset}: A commonly used dataset in automatic rule generation for 10-class classification, with 1 million instances and 25 features. Uses multivariate logistic loss.
  \end{itemize}
  \textit{These datasets, with different tasks (binary vs. multi-class) and domains, test the algorithm's scalability and generalization.}
  
  \vspace{0.2cm}
  \textbf{Comparison and Verification Goals}:
  \begin{itemize}
    \item Compare AD-OGP with D-OGP (synchronous) to show the advantage of asynchronization.
    \item Verify the effectiveness of asymmetric gossiping and instantaneous model averaging.
    \item Verify our theoretical regret bound.
  \end{itemize}
  
  \vspace{0.2cm}
  \textbf{Network Topologies for Comparison}:
  \begin{itemize}
    \item Complete graph (high connectivity)
    \item Watts-Strogatz graph (random graph with medium connectivity)
    \item Ring graph (low connectivity)
  \end{itemize}
\end{frame}



\begin{frame}
  \frametitle{The Benefit of Asynchronization}
  \begin{figure}[h!]
    \centering
    \maybeincludegraphics[width=0.8\textwidth]{image4.png}
    \caption{Illustration of the benefit of asynchronization. The four plots compare AD-OGP and D-OGP on 64-node Watts-Strogatz graphs on poker-hand under varying delay levels.}
  \end{figure}
  \begin{itemize}
    \item AD-OGP's running speed is significantly faster than D-OGP.
    \item The performance gap increases with higher processing and message delays.
    \item AD-OGP's average loss is comparable to D-OGP, indicating no sacrifice in learning performance.
  \end{itemize}
\end{frame}

\begin{frame}
  \frametitle{The Effectiveness of Asymmetric Gossiping}
  \begin{figure}[h!]
    \centering
    \maybeincludegraphics[width=0.8\textwidth]{image5.png}
    \caption{Illustration of the effectiveness of asymmetric gossiping.}
  \end{figure}
  \begin{itemize}
    \item Increasing $D^{msg}$ artificially creates communication bottlenecks.
    \item Experiments show that asymmetric gossip communication significantly improves performance, especially prominent with high communication delays.
  \end{itemize}
\end{frame}

\begin{frame}
  \frametitle{The Effectiveness of Instantaneous Model Averaging}
  \begin{figure}[h!]
    \centering
    \maybeincludegraphics[width=0.8\textwidth]{image6.png}
    \caption{Illustration of the effectiveness of instantaneous model averaging.}
  \end{figure}
  \begin{itemize}
    \item Adjusting $D_l$ magnifies learner waiting time.
    \item Experiments demonstrate that instantaneous model averaging significantly boosts performance by utilizing idle time, especially when noticeable idle periods exist.
  \end{itemize}
\end{frame}

\begin{frame}
  \frametitle{Verification of the Theoretical Bound}
  \begin{figure}[h!]
    \centering
    \maybeincludegraphics[width=0.8\textwidth]{image7.png}
    \caption{Verification of our regret bound. The four plots show AD-OGP’s performance on poker-hand with varying characteristics. (a) Varying network sizes. (b) Varying network topology. (c) Varying levels of processing delays. (d) Varying levels of message delays.}
  \end{figure}
  \begin{itemize}
    \item Experiments adjust four factors in the regret bound formula across the four sub-figures.
    \item Results indicate that AD-OGP's loss is higher when:
    \begin{itemize}
      \item The number of learners is larger.
      \item Network connectivity is worse.
      \item Processing delays are heavier.
      \item Message delays are heavier.
    \end{itemize}
    \item This shows a high consistency between experimental behavior and theoretical results.
  \end{itemize}
\end{frame}

\section{Conclusion}
\begin{frame}
  \frametitle{Conclusions}
  \textbf{Key Contributions}:
  \begin{itemize}
    \item In this paper, we present the first systematic study of asynchronous decentralized online learning.
    \item We begin by formulating the framework of Asynchronous Decentralized Online Convex Optimization (AD-OCO), which gives a complete characterization of the whole interaction dynamics.
    \item Then we devise the Asynchronous Decentralized Online Gradient-Push (AD-OGP) algorithm, which is fully asynchronous and includes two novel innovations: weighted projection and instantaneous model averaging.
    \item Theoretically, we provide the first regret analysis paradigm for asynchronous decentralized online learning, deriving a non-trivial regret bound of $O(\sqrt{T})$.
    \item Finally, we conduct extensive experiments to demonstrate the benefit of asynchronization, verify the effectiveness of the two innovations, and corroborate the theoretical bound.
    \item Our work paves the way for future research investigating asynchronous decentralized online learning.
  \end{itemize}
\end{frame}


\begin{frame}
  \frametitle{Limitations and Future Work}
  \textbf{Limitations and Future Work}:
  \begin{itemize}
    \item As the first step of studying asynchronous learning in the decentralized online setting, our framework formulation and theoretical analysis are conducted in the convex setting.
    \item Although such limitation does not affect its usage in the non-convex setting, we would like to establish a formal non-convex analysis in the future.
    \item Further explore more complex network scenarios and dynamic topologies.
  \end{itemize}
\end{frame}

\section{References}
\begin{frame}{References}
\begin{thebibliography}{99}
\bibitem{adoco1} Anonymous Author 1, Anonymous Author 2, Anonymous Author 3. (2023).\\
{\bf Asynchronous Decentralized Online Convex Optimization.\\}
In Submission to Major Machine Learning Conference.

\bibitem{decentralized1} Tsianos, K. I., Lawlor, S., \& Rabbat, M. G. (2012).\\
{\bf Consensus-based distributed optimization: Practical issues and applications in large-scale machine learning.\\}
In 2012 50th Annual Allerton Conference on Communication, Control, and Computing.

\bibitem{pushsum1} Kempe, D., Dobra, A., \& Gehrke, J. (2003).\\
{\bf Gossip-based computation of aggregate information.\\}
In 44th Annual IEEE Symposium on Foundations of Computer Science.
\end{thebibliography}
\end{frame}

\end{document}
